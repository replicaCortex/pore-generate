\hl{УДК 004.932.2:666.982}

СТОХАСТИЧЕСКАЯ ГЕНЕРАЦИЯ ИЗОБРАЖЕНИЙ ПОРИСТЫХ МАТЕРИАЛОВ С ЗАДАННЫМИ
МОРФОЛОГИЧЕСКИМИ ХАРАКТЕРИСТИКАМИ ДЛЯ ОБУЧЕНИЯ НЕЙРОННЫХ СЕТЕЙ

\emph{\textbf{О.С. Тушканова, С.С. Беликов}}

Южно-Российский государственный политехнический университет (НПИ),
\href{mailto:lynx77@inbox.ru}{\emph{lynx77@inbox.ru}},
\href{mailto:replicacortex@gmail.com}{\emph{replicaCortex@gmail.com}} --
твой email

\emph{\textbf{Аннотация.} \hl{Предложен новый метод генерации
синтетических изображений пористых материалов для обучения нейросетей
сегментации. Основанный на алгоритмах Монте-Карло, метод создает
бинарные маски и соответствующие grayscale-изображения, обеспечивая
статистическую достоверность микроструктур при детерминированном
контроле пористости, распределения пор по размерам и их
пространственного расположения. Подход позволяет создавать
неограниченные объемы обучающих данных на основе ограниченного числа
эталонных образцов, эффективно решая проблему дефицита размеченных
данных.}}

\emph{\textbf{Ключевые слова:}~семантическая сегментация, генерация
синтетических данных, пористые материалы, морфология пор, метод
Монте-Карло, нейронные сети, геополимеры.}

STOCHASTIC GENERATION OF POROUS MEDIA IMAGES WITH CONTROLLED
MORPHOLOGICAL CHARACTERISTICS FOR NEURAL NETWORK TRAINING

\emph{\textbf{O.S. Tushkanova, S.S. Belikov}}

South-Russian State Polytechnic University
(NPI),~\href{mailto:lynx77@inbox.ru}{\ul{lynx77@inbox.ru}},
\href{mailto:replicacortex@gmail.com}{\ul{replicaCortex@gmail.com}}

\emph{\textbf{Abstract.}} This paper proposes a novel method for
generating synthetic images of porous materials to train neural networks
for segmentation. Based on Monte Carlo algorithms, the method produces
binary masks and corresponding grayscale images while ensuring
statistical reliability of microstructures and deterministic control
over porosity, pore size distribution, and spatial arrangement. The
approach enables generating unlimited training data volumes from limited
reference samples, effectively solving the problem of annotated data
scarcity..

\emph{\textbf{Keywords:}}~\emph{semantic segmentation, synthetic data
generation, porous materials, pore morphology, Monte Carlo method,
neural networks, geopolymers.}

\subsection{Введение}\label{ux432ux432ux435ux434ux435ux43dux438ux435}

Для решения задач семантической сегментации микроструктуры пористых
материалов широко применяются нейронные сети. Однако их эффективность
напрямую зависит от объема и качества обучающих данных {[}1{]}. Создание
репрезентативных размеченных наборов данных сопряжено с значительными
трудностями, обусловленными высокой стоимостью экспериментов,
трудоемкостью ручной разметки изображений и ограниченным доступом к
специализированному оборудованию.

Хотя проблему нехватки данных часто пытаются решить с помощью
генеративных нейронных сетей, для научного анализа ключевое значение
имеет не только визуальное правдоподобие, но и статистическая
достоверность генерируемых структур {[}2{]}.

В случае решения задачи сегментации пористых геополимеров генеративные
модели не позволяют контролировать ключевые морфологические параметры:
пористость, а также распределение пор по размерам и в пространстве.

Под визуальным правдоподобием в данном контексте понимается схожесть
синтетического изображения с реальным на качественном уровне. В то время
как статистическая достоверность подразумевает количественное совпадение
ключевых параметров, таких как общая пористость, распределение пор по
размерам и их морфологические характеристики (например, эксцентриситет,
округлость).

Целью данной работы является разработка метода, который генерирует
статистически достоверные изображения элементов обучающей выборки,
представляющих собой пары «изображение в градации серого -- бинарная
маска» (рисунок 1), с помощью алгоритмов Монте-Карло.

\includegraphics[width=2.52326in,height=2.46389in]{media/image4.png}\includegraphics[width=2.36047in,height=2.47674in]{media/image2.png}

а) б)

\textbf{Рисунок 1 --} \textbf{Обучающая выборка для задачи сегментации:}

а) изображения в градации серого; б) бинарные маски

Метод должен строго контролировать параметры пористости и
морфологические параметры пор, чтобы полученные изображения можно было
использовать для обучения нейросетевых моделей сегментации.

\subsection{Категории и формы
пор}\label{ux43aux430ux442ux435ux433ux43eux440ux438ux438-ux438-ux444ux43eux440ux43cux44b-ux43fux43eux440}

Источником данных для генерации бинарных изображений служит эталонная
маска, содержащая изображения пор (пора -- черный пиксель, матрица или
фон -- белый пиксель).

Поры классифицируются на три категории по размеру: крупные, средние и
малые.

По типу взаимного расположения поры подразделяются на: изолированные
(автономные), касающиеся (площадь пересечения до 5\%), пересекающиеся
(площадь пересечения 5-15\%) и образующие кластеры (две и более пор с
площадью пересечения свыше 15\%).

Для генерации изображений в градациях серого используются параметры,
извлеченные из эталонного изображения: минимальная и максимальная
яркость пикселей, принадлежащих матрице, и минимальная и максимальная
яркость пикселей, принадлежащих порам.

\subsection{Формат описания
параметров}\label{ux444ux43eux440ux43cux430ux442-ux43eux43fux438ux441ux430ux43dux438ux44f-ux43fux430ux440ux430ux43cux435ux442ux440ux43eux432}

Для управления процессом генерации и сохранения геометрических
характеристик пор каждой маски используются два типа JSON-файлов:
конфигурационный файл и файл с метаданными.

Конфигурационный файл создается на основе анализа бинарного и серого
изображений. (Примечание. Анализ эталонной маски в задачу не входит,
считается заданным для описываемого метода.) Конфигурационный файл
(JSON\_1) описывает целевые параметры генерируемой выборки (рисунок 2).

Для генерации бинарной маски: в нем задаются размер бинарного
изображения (холста), количество пор каждой категории и их отклонения
(±1-2\% от общего количества), размеры пор каждой категории и их
отклонения, диапазон параметров деформации поры (коэффициента
сжатия-растяжения), минимальные расстояния между порами разных
категорий, допустимое отклонение статистических параметров (например,
±5\% для общей пористости).

Для генерации изображения в градациях серого по созданной бинарной маске
задаются диапазоны яркостей пор и фона, а также параметры процедурного
шума для придания реалистичности.

\begin{longtable}[]{@{}
  >{\raggedright\arraybackslash}p{(\linewidth - 2\tabcolsep) * \real{0.4999}}
  >{\raggedright\arraybackslash}p{(\linewidth - 2\tabcolsep) * \real{0.5001}}@{}}
\toprule\noalign{}
\begin{minipage}[b]{\linewidth}\raggedright
\{\\
"image\_settings": \{\\
"width": 200,\\
"height": 200,\\
"total\_images": 10\\
\},\\
"pore\_settings": \{\\
"large\_pores": \{\\
"count\_range": {[}3, 5{]},\\
"radius\_mean": 25,\\
"stretch\_factor\_range": {[}1, 1.1{]}\\
\},\\
"medium\_pores": \{\\
"count\_range": {[}10, 15{]},\\
"radius\_mean": 12,\\
"stretch\_factor\_range": {[}1, 1.1{]}\\
\},\\
"small\_pores": \{\\
"count\_range": {[}20, 30{]},\\
"radius\_mean": 6\\
\}\\
\},\strut
\end{minipage} & \begin{minipage}[b]{\linewidth}\raggedright
\begin{quote}
"noise\_settings": \{\\
"matrix\_gray\_range": {[}100, 200{]},\\
"pore\_gray\_range": {[}0, 100{]},\\
"noise\_intensity": 0.1\\
"gaussian\_sigma ": 1.5\\
\}\\
"min\_gray\_value": 180,\\
"max\_gray\_value": 230,\\
"min\_pore\_value": 20,\\
"max\_pore\_value": 70,\\
""gaussian\_sigma": \{\\
"description": "Стандартное отклонение Гауссова ядра для размытия",\\
"typical\_range": 1.5,\\
"common\_values": \{\\
"subtle\_smoothing": 0.3,\\
"natural\_look": 0.8,\\
"strong\_blur": 1.3,\\
"high\_resolution": 0.3,\\
"low\_resolution": 1.5\\
\}\\
\}
\end{quote}\strut
\end{minipage} \\
\midrule\noalign{}
\endhead
\bottomrule\noalign{}
\endlastfoot
\end{longtable}

\textbf{Рисунок 2} --\textbf{~Структура конфигурационного файла}

Файл с характеристиками пор (JSON\_2) содержит детальную информацию о
каждой сгенерированной поре. Для последующего анализа и контроля в файл
метаданных для каждой поры записываются координаты ограничивающего
прямоугольника (\emph{bounding box}), а также ее геометрические
характеристики в двух состояниях: исходном (\emph{original}) и
деформированном (\emph{deformed}). В разделе~original~хранятся параметры
идеальной сферической поры: радиус и площадь. В разделе \emph{deformed}
содержатся параметры видоизмененной «реальной» поры: новый радиус,
площадь, эксцентриситет и округлость в диапазоне {[}0..1{]}.

Фрагмент JSON\_2 приведен на рисунке 3.

\begin{quote}
\{\\
"large\_pores": {[}\\
\{\\
"box": \{"x\_min": 50, "y\_min": 40, "x\_max": 88, "y\_max": 82\},\\
"original": \{\\
"radius": 28,\\
"area": 2463\\
\},\\
"deformed": \{\\
"radius": 27,\\
"area": 2335,\\
"eccentricity": 0.2469,\\
"circularity": 0.7408\\
\}\\
\}\\
{]}\\
\}
\end{quote}

\textbf{Рисунок 3 - Пример фрагмента JSON-файла с метаданными для
автономной поры}

\subsection{Описание
алгоритма}\label{ux43eux43fux438ux441ux430ux43dux438ux435-ux430ux43bux433ux43eux440ux438ux442ux43cux430}

Алгоритм процедурно генерирует синтетические изображения, используя
стохастический подход, основанный на методе Монте-Карло. Процесс состоит
из двух основных этапов: этапа генерации и размещения пор (создание
бинарной маски; этапа формирование изображения в градациях серого.

На первом этапе:

\begin{enumerate}
\def\labelenumi{\arabic{enumi}.}
\item
  Задается радиус «идеальной» (сферической) поры на основе распределения
  Пуассона в соответствии с конфигурацией.
\item
  Рассчитывается площадь идеальной поры.
\item
  На пору накладывается правильный многоугольник со случайным числом
  сторон для нарушения идеальной формы.
\item
  Применяются процедуры сглаживания для устранения артефактов.
\item
  К полученной форме применяются аффинные преобразования: случайное
  растяжение/сжатие по осям и поворот.
\item
  Характеристики полученной поры заносятся в раздел \emph{deformed}
  файла JSON\_2. Таким образом формируется количество пор каждой
  категории, заданное в конфигурационном файле с учетом допустимых
  отклонений.
\end{enumerate}

\includegraphics[width=1.61739in,height=1.48403in]{media/image5.png}

\includegraphics[width=1.45625in,height=1.44028in]{media/image6.png}\includegraphics[width=1.3913in,height=1.33959in]{media/image3.png}\includegraphics[width=1.61811in,height=1.35039in]{media/image1.png}

а) б) в) г)

\textbf{Рисунок 4} -- \textbf{Иллюстрация процесса формирования реальной
поры:}

а) идеальная пора; б) совмещение поры с многоугольником; в) сглаживание
мест стыковки и углов; г) растяжение и поворот поры.

\begin{enumerate}
\def\labelenumi{\arabic{enumi}.}
\setcounter{enumi}{6}
\item
  Далее алгоритм Монте-Карло размещает сгенерированные поры на холсте,
  начиная с крупных пор и заканчивая малыми. Для каждой поры алгоритм
  выполняет множественные попытки найти подходящую позицию, контролируя
  коллизии (пересечение пор) и соблюдая минимальное расстояние между
  порами, заданное в конфигурационном файле. Если разместить пору не
  удается в течение 200 попыток, она отвергается. Координаты
  ограничивающего прямоугольника установленной поры заносятся в JSON\_2.
\item
  После завершения процесса размещения осуществляется валидация
  сгенерированной маски. Если отклонение параметров пористости или
  количество установленных пор превышают заданную точность, маска
  отвергается.
\end{enumerate}

На втором этапе:

\begin{enumerate}
\def\labelenumi{\arabic{enumi}.}
\item
  Генерация фоновой текстуры матрицы. Исходный холст заполняется
  синтетической текстурой, имитирующей микроструктуру материала. Для
  создания статистически достоверной неоднородности используется
  процедурная генерация на основе многослойного фрактального шума
  (Фурье-спектральный синтез). Данный подход предполагает суперпозицию
  нескольких октав шума Перлина с различными пространственными частотами
  и амплитудами, что позволяет воспроизводить характерные для
  геополимеров флуктуации плотности. Интенсивность пикселей матрицы
  нормализуется в соответствии с параметрами конфигурационного файла
  JSON\_1 (\emph{min\_gray\_value}, \emph{max\_gray\_value}),
  обеспечивая соответствие эталонным данным.
\item
  Моделирование морфологии пор. Для каждой поры, координаты и геометрия
  которой определены на этапе генерации бинарной маски, выполняется
  процедура текстурирования. Область поры заполняется фрактальным шумом
  с параметрами, отличными от фоновых, что позволяет воспроизводить
  внутреннюю неоднородность pore structure. Диапазон яркости пикселей
  поры задаётся параметрами \emph{min\_pore\_value} и
  \emph{max\_pore\_value} из JSON-конфигурации. Особое внимание
  уделяется плавному переходу на границе раздела матрица-пора, где
  применяется градиентное смешивание интенсивностей.
\item
  Постобработка и сглаживание. Для финальной интеграции сгенерированных
  структур и повышения фотореалистичности применяется пространственная
  фильтрация двумерным Гауссовым ядром с малым радиусом (σ = 0.5-1.5
  пикселя). Данная операция позволяет:
\end{enumerate}

\begin{itemize}
\item
  нивелировать дискретность процедурно сгенерированных текстур;
\item
  имитировать естественное размытие, присутствующее в реальных
  микрофотографиях;
\item
  обеспечить плавные переходы на границах раздела фаз.
\end{itemize}

\hl{Степень размытия контролируется набором параметров}
\emph{gaussian\_sigma} \hl{в конфигурационном файле, что позволяет
адаптировать характеристики выходного изображения под различные
экспериментальные условия съёмки.}

Главное преимущество алгоритма заключается в детерминированном
контролировании морфологических параметров пористого материала, сохраняя
при этом стохастическую природу генерации. Это позволяет создавать
статистически достоверные синтетические структуры с точно заданными
характеристиками.

\subsection{Заключение}\label{ux437ux430ux43aux43bux44eux447ux435ux43dux438ux435}

Предложенный метод генерации статистически достоверных синтетических
изображений пористых материалов на основе алгоритмов Монте-Карло
позволяет создавать обучающие выборки сколь угодно больших размеров на
основе малого числа эталонных пар «серое изображение -- бинарная маска»
с сохранением статистической точности, решая проблему дефицита данных
для обучения нейросетей.

Дальнейшее совершенствование алгоритма планируется вести по двум
направлениям: 1) учет эффекта обрезки пор на границах эталонных
изображений; 2) более достоверная «раскраска» изображений в градациях
серого.

\textbf{Список используемых источников}

\begin{enumerate}
\def\labelenumi{\arabic{enumi}.}
\item
  Иванов, А.А.~Генерация синтетических данных для сегментации
  медицинских изображений с использованием условных generative
  adversarial networks / А.А. Иванов, С.В. Петров // Искусственный
  интеллект и принятие решений. -- 2023. -- № 2. -- С. 45-58. -- DOI:
  10.14357/20718594230204.
\item
  Сидорова, М.К.~Методы Монте-Карло в задачах компьютерного
  моделирования пористых сред / М.К. Сидорова, П.Д. Федоров //
  Компьютерные исследования и моделирование. -- 2022. -- Т. 14, № 4. --
  С. 789-802. -- DOI: 10.20537/2076-7633-2022-14-4-789-802.
\end{enumerate}
